\documentclass[a4paper,10pt]{article}
\usepackage[utf8x]{inputenc}
\usepackage{array}
\usepackage{graphicx}
\usepackage{float}
\usepackage{color}
\usepackage{alltt}
\usepackage[T1]{fontenc}
\usepackage{ae,aecompl}

%opening
\title{General Context Free Top Down Parsing in Cubic Time}
\author{Arnold Lankamp}

\begin{document}

\maketitle

\begin{abstract}

\end{abstract}

\section{Introduction}

Highlight advantages.\\
Simple, scalable, fast, good error messages, $O(N^3)$ with input, $O(N)$ with grammar size, linear with most unambiguous grammars, deterministic with LL and LR, easy to visualize, easy to generate good error messages and so on ....

\section{Recognizer}

Discuss recognizer first, then parser.

\subsection{Parallel LL}

Split stack for each alternative.

\subsection{Sharing}

Fix unbound polynomial behaviour.\\
Fix LL shortcomings.

\subsubsection{Graph}

GSS like.\\
Each node has an id, edges and an optional next node.

\subsubsection{Left recursion}

Sharing fixes the looping behaviour of LL.

\subsubsection{Cycles}

Sharing fixes cycle looping.

\subsubsection{Nullables}

We need multiple iterations in the same level, which complicates sharing.

\subsubsection{Hidden right recursion}

Propagate node updates forwards and execute reductions where needed.

\subsubsection{Termination}

Move to the next level when done.\\
Each level converges to completion, due to sharing.\\
Assumption: Input isn't infinite.

\subsubsection{Breadth-first vs depth-first}

Recursive decent is possible too.\\
Doesn't break the cubic time bound.\\
Node visits may be $N$ higher, but work per visit is inversely proportional to the amount of visits (1 edge added per time now).

\subsection{Worst-case time complexity}

$O(N^3)$ at most.

\subsection{Unambiguous case time complexity}

$O(N^2)$ upper bound.\\
Deterministic on all classes of LL and LR grammars; depending on look-ahead.\\
$O(N)$ on split only.\\
$O(N^2)$ on split + merge.\\
Graceful performance degradation with ambiguous grammars.\\
At least as good as others.

\subsection{Psuedocode}

Expand.\\
Reduce.\\
Main loop.

\section{Parser}

Recognizer simple to modify.\\
Will explain how.\\
But talk about forest representation first.

\subsection{Parse forest}

Parse forest must be cubic to maintain $O(N^3)$ scalability guarantee.\\
Has results, prefixes and nodes.

\subsection{Recognizer augmentation}

Easy.\\
Add prefixes to graph nodes.\\
Create results during reductions.

\subsubsection{Cycles}

Cycle in Graph, means cycle in forest as well; so no problem.

\subsubsection{Hidden right recursion}

Propagate prefixes properly as well.

\subsubsection{Psuedocode}

Expand (same).\\
Reduce.\\
Main loop.\\
Result creation.

\subsection{Worst-case complexity}

Forest is $O(N^3)$ in size at most.

\section{Optimizations}

Make the performance acceptable.

\subsection{Look-ahead}

Makes LL and LR grammars deterministic.

\subsection{Edge optimizations}

Sharing needs caring.

\subsubsection{Expansion}

Makes performance linear with respect to the size of the grammar, regardless of what it looks like (i.e. recursion and such).

\subsubsection{Reduction}

Lowers edge visits to a 'perfect' level.

\subsection{Prefix sharing}

Makes performance linear for LR grammars.

\section{Benchmarks}

Show off.

\subsection{Worst-case}

Fast $O(N^3)$ parsing.

\subsubsection{Statistics}

Worst-case stats for graph and forest.

\subsection{Grammar factoring}

Linear all around.

\subsection{Realistic cases}

C, Java and such would be nice.\\
Sucks to compare them to other parsers though.

\section{Discussion}

We're better and so on ....

\section{Related work}

GLL I suppose.\\
Anything else?

\section{Future work}

\begin{itemize}
 \setlength{\itemsep}{0pt}
 \setlength{\parskip}{0pt}
 \setlength{\parsep}{0pt}
 
 \item Filtering
 \item More optimizations
\end{itemize}

\section{Conclusion}

Fast, scalable, simple, good error messages, easy to visualize, perfect ....

\section{Acknowledgement}

TODO

\section{References}

Nah, did it all myself ;-).

\end{document}
